%----------------------------------------------------------------------------------------
%	PACKAGES AND OTHER DOCUMENT CONFIGURATIONS
%----------------------------------------------------------------------------------------

\documentclass[11pt]{scrartcl} % Font size

\input{structure.tex} % Include the file specifying the document structure and custom commands

\usepackage{glossaries}
% Make glossaries
\makenoidxglossaries

% Makes the text of glossary entries italic
\renewcommand{\glstextformat}[1]{\textit{#1}}

% Glossary entries
\newglossaryentry{ExtJs}
{
    name=ExtJs,
    description={ExtJS is a JavaScript framework for building web applications with a rich user interface}
}

\newglossaryentry{Hadoop}
{
    name=Hadoop,
    description={Hadoop is a distributed processing framework written in Java that allows for large-scale data storage and processing}
}

\newglossaryentry{Nodejs}
{
    name=Nodejs,
    description={Node.js is an open-source, cross-platform JavaScript runtime environment that enables server-side scripting}
}

\newglossaryentry{Expressjs}
{
    name=Expressjs,
    description={Express.js is a lightweight and flexible web application framework for Node.js used for building scalable and efficient server-side applications~\parencite{expressjs}}
}

\newglossaryentry{Postgresql}
{
    name=Postgresql,
    description={PostgreSQL is a powerful, open-source object-relational database system that provides advanced features for scalability, reliability, and data integrity}
}

\newglossaryentry{Reactjs}
{
    name=Reactjs,
    description={React.js is a JavaScript library that allows the creation of interactive user interfaces using reusable UI components}
}

\newglossaryentry{ReCharts}
{
    name=ReCharts,
    description={ReCharts is a lightweight, open-source, charting library that provides a wide range of customizable charts and graphs for data visualization}
}

\newglossaryentry{Weather API}
{
    name=Weather API,
    description={Weather API is an API provided by weatherapi.com that provides real-time weather data for a wide range of locations and parameters}
}

\newglossaryentry{Python}
{
    name=Python,
    description={Python is a high-level, interpreted programming language with dynamic semantics that is widely used in various fields, including data science, machine learning, and web development}
}

\newglossaryentry{Tensorflow}
{
    name=Tensorflow,
    description={TensorFlow is an open-source machine learning framework developed by Google for building and training neural networks}
}

% Acronyms
% \newacronym{cms}{CMS}{Conten Management System}

 % Include the file specifying the glossary / acronym entries

\addbibresource{main.bib}

%----------------------------------------------------------------------------------------
%	TITLE SECTION
%----------------------------------------------------------------------------------------

\title{	
	\normalfont\normalsize
	\textsc{Fontys University of Applied Sciences \\A research conducted for the minor A-Systems}\\ % Your university, school and/or department name(s)
	\vspace{25pt} % Whitespace
	\rule{\linewidth}{0.5pt}\\ % Thin top horizontal rule
	\vspace{20pt} % Whitespace
	{\huge IoT soil moisture monitoring to predict irrigation needs of outdoor crops}\\ % The assignment title
	\vspace{12pt} % Whitespace
	\rule{\linewidth}{2pt}\\ % Thick bottom horizontal rule
	\vspace{12pt} % Whitespace
}

\author{\LARGE Niklas Mezynski} % Your name

\date{\normalsize\today} % Today's date (\today) or a custom date

\begin{document}

\maketitle % Print the title

% Remove page number from title page
\thispagestyle{empty}

\newpage


%----------------------------------------------------------------------------------------
%	ToC and page/section numbering
%----------------------------------------------------------------------------------------

\pagenumbering{roman}

% Table of contents depth of contents
\setcounter{tocdepth}{3}
\tableofcontents

\newpage
\listoffigures

\newpage
\listoftables

\clearpage
\printnoidxglossaries
\cleardoublepage
\pagenumbering{arabic}
\setcounter{section}{1}

%----------------------------------------------------------------------------------------
%	BEGIN CONTENT
%----------------------------------------------------------------------------------------

\subsection{Introduction}
\subsubsection{Background and context}
Soil moisture monitoring is a crucial aspect of modern agriculture. The accurate monitoring of soil moisture levels can improve crop yields, save water, and increase efficiency. Currently, many farmers use manual methods to monitor soil moisture levels, which can be time-consuming and imprecise. In recent years, IoT sensors have been developed, which can provide real-time data on soil moisture levels~\parencite{bwambale2022smart}.
\newline Although there are various sensors already being used in modern agriculture, it is not clear which sensor is the most accurate and effective for different soil types, crop types, and environmental conditions. Additionally, the integration of weather data in irrigation scheduling is often not being considered~\parencite{nandurkar2014design}.
\newline Moreover, the traditional manual methods of monitoring soil moisture levels and irrigation scheduling are labor-intensive, time-consuming, and often based on assumptions rather than actual data. This results in excessive water usage, which not only wastes a valuable resource but also impacts the environment negatively~\parencite{bwambale2022smart}.
\newline Therefore, the lack of an accurate and efficient system for monitoring soil moisture levels and predicting crop irrigation needs results in a significant gap in knowledge in the agricultural industry. Through this research project, the aim is to develop a cloud-based application that can accurately monitor soil moisture levels, integrate weather data, and predict crop irrigation needs.

\subsubsection{Research questions and objectives}
This study's research question seeks to investigate the water-saving potential of a cloud-based application that integrates real-time soil moisture and weather data in predicting the irrigation needs of outdoor crops, as well as how it compares to traditional manual irrigation scheduling methods. The study has identified several theoretical learning objectives to accomplish this.
\newline Understanding the principles of IoT sensors and their applications in agriculture, the basics of cloud computing and monitoring and their applications in agriculture, exploring algorithms for predicting crop irrigation needs using soil moisture and weather data, finding a correlation between certain weather data parameters and changes in soil moisture levels to make predictions, and developing a cloud-based monitoring application that can make a decision are some of the objectives. The study's goal is to shed light on the efficacy of cloud-based applications and how they may be utilized to enhance crop irrigation, ultimately contributing to the development of sustainable and efficient agricultural practices.
\subsubsection{Overview of the experiment and methods}
The experiment will take place in an open field-like setting. The experiment will be divided into two phases. In the first phase, soil moisture levels will be monitored and data is collected using IoT sensors. In the second phase, the collected data will be used to develop a cloud-based application that can predict irrigation needs for outdoor crops. The experiment will take place over the course of four weeks. The first two weeks will be spent collecting data, and the following two weeks will be spent verifying the algorithm and its efficiency. The experiment will be carried out as follows:
\begin{itemize}
	\item The soil moisture levels will be monitored using IoT sensors.
	\item The data will be collected and stored in a database.
	\item The data will be analyzed to find a correlation between certain weather data parameters and change in soil moisture levels.
	\item The data will be used to develop a cloud-based application that can predict irrigation needs of outdoor crops.
	\item The algorithm will be verified and its efficiency will be tested.
\end{itemize}

\subsection{Literature review}
\subsubsection{IoT sensors and their applications in agriculture}
Agriculture is facing huge challenges as a result of population growth and climate change. As the agricultural sector is responsible for 71\% of global water withdrawals, it is critical to establish sustainable techniques in order to meet the demands for food production. IoT sensors, for example, have the potential to change agriculture and address these difficulties. The use of IoT sensors and wireless sensor networks into agriculture has created new prospects for precision farming and resource optimization.~\parencite{capacitive_sensors_for_irrigation_and_disease, worldwide_water_withdrawals}
\newline A smart irrigation system was presented in one study that used IoT and machine learning to offer correct irrigation to plants. The technology lowered water consumption by 46\% while enhancing plant health. Another study built a low-power and scalable IoT-based architecture to monitor soil moisture and temperature for home farmers and scientific applications. The measured data was sent to a cloud platform for processing, providing home farmers with critical information to improve crop efficiency and reduce crop failure. The study's hardware and software architecture were designed to require less power, making it an efficient and cost-effective solution.~\parencites{precision_irrigation_iot}{capacitive_sensors_for_irrigation_and_disease}
\newline The application of IoT sensors in agriculture has also demonstrated promise in smart farming and precision agriculture. Sensors can be used to detect illnesses, test NPK levels, and detect soil moisture content, among other things. Smart farming optimizes traditional farming processes by utilizing precise information to make informed decisions. Control and supervision, safety, warning, diagnostics, and analytics are all applications of IoT sensors in agriculture, making it more effective and trouble-free.~\parencite{iot_and_sensors_in_agriculture_general}
\newline Finally, IoT sensors provide several benefits to agriculture, including reduced water usage, increased efficiency, fewer crop failures, and improved yield. As the world population expands and food demand rises, the use of IoT and sensor networks in agriculture will become even more vital in order to provide food for everyone in a sustainable manner.\parencite{capacitive_sensors_for_irrigation_and_disease,bwambale2022smart}
\newline There could be numerous reasons for the low number of real-world uses of soil moisture sensors in open fields. One issue is the high cost and complexity of sensor installation and maintenance, which can be too expensive for many farmers. Furthermore, there may be a lack of understanding and education regarding the benefits of employing soil moisture sensors, as well as the best practices for using them. Finally, regulatory and policy constraints may exist that limit the deployment of certain technologies in specific regions or countries. More study is required to overcome these issues and promote the wider use of IoT sensors for sustainable agriculture practices.

\subsubsection{Cloud computing and monitoring in agriculture}
A 2019 study explores the development of a networking application system for modern agriculture with the goal of addressing agricultural product quality safety and environmental contamination. Their solution is built on open-source hardware, which makes video surveillance based on motion detection easier. In the development and implementation of a demo system, the authors created a basic cloud platform system for modern farm network monitoring that employs a RESTful interface service system and ExtJs client technology. The Hadoop platform is utilized to achieve huge data processing generated by Internet of Things applications, and the associated model is developed using machine learning technology. The research addresses crop variety selection, production and cultivation management, and time to market.~\parencite{cloud_computing_and_monitoring}
\newline Currently, Internet of Things-related technologies and products are becoming more mature, and the associated software and hardware facilities, technical standards, and transmission protocols are being gradually upgraded. These mature hardware and software solutions are used in agricultural production for data collection, transmission, storage, processing, feedback, and control.~\parencite{cloud_computing_and_monitoring}
\newline The study focuses on the advantages of cloud computing in the Internet of Things. The cloud platform provides an easy sensing data transmission interface for the sensing device, and the user can easily transmit data acquired by the sensor to the cloud platform. The cloud platform stores, analyzes, and processes data, as well as providing informative feedback based on user-defined decision rules. Users can also use the platform for equipment administration, data inquiry, visualization, warning, information push, and other operations via a web interface or mobile app. The cloud services platform improves the security, stability, and low-cost operation of the necessary applications, allowing developers to focus on application development.~\parencite{cloud_computing_and_monitoring}
\subsubsection{Biological Aspects of Soil Moisture and Crop Irrigation Needs}
\paragraph{The effect of soil moisture on plant growth and yield}
Soil moisture is an important aspect of plant growth and development. The relationship between soil moisture and crop development, yield, and quality is detailed in this section. Also covered is the significance of available moisture for irrigation control, nutrient uptake, and drought management.~\parencite{soil_moisture_plant_water}
\newline Soil moisture is influenced by a variety of factors, including soil type, climate, rainfall, irrigation, and soil management practices. The texture, structure, and organic matter composition of the soil determine its water-holding ability. Understanding these elements is critical for good soil moisture management.~\parencite{soil_moisture_plant_water}
\newline The maximum amount of water that soil can hold after surplus water has been drained away and the rate of downward flow has been reduced is referred to as field capacity. When the soil is at \textbf{field capacity}, it can hold the maximum quantity of water available to plants without becoming waterlogged or generating anaerobic conditions. The soil moisture content at which plants can no longer take water from the soil, causing wilting and eventually death if the water supply is not restored, is known as the \textbf{permanent wilting point}.~\parencite{soil_moisture_plant_water}
\newline Soil moisture has an impact on several aspects of agriculture. One being the impact on crop growth and yield. Adequate soil moisture is necessary for optimal crop growth and development, which directly impacts crop yield and quality. Insufficient soil moisture can lead to plant stress, reduced growth, and lower yields, while excessive soil moisture can cause waterlogging, root rot, and other issues.~\parencite{soil_moisture_plant_water}
\newline Knowing the available moisture in the soil also allows for more efficient irrigation scheduling and water management. By ensuring that the soil moisture stays within the range between field capacity and the permanent wilting point, farmers can optimize water use, prevent over- or under-watering, and promote healthy plant growth.~\parencite{soil_moisture_plant_water}
\newline Soil moisture is also important for the availability and uptake of nutrients by plants. Many nutrients dissolve in soil water and are absorbed by plant roots via the water uptake process. Farmers can ensure that plants have access to vital nutrients for growth and development by maintaining appropriate soil moisture levels.~\parencite{soil_moisture_plant_water}
\newline In conclusion, soil moisture is an important factor in crop growth, yield, and quality. Farmers must effectively manage soil moisture in order to maximize crop production and ensure sustainable agriculture.
% \subsubsection{Physiological responses of plants to water stress}
% \subsubsection{Advantages and disadvantages of using soil moisture as an indicator for crop irrigation needs}


\subsection{Experiment}
\subsubsection{Design and setup}
The first stage of the experiment is dedicated to data collection. For the experiment, six salad crops are planted in two pots (one for testing the algorithm and one for comparison in the second stage) that are set outside in the open air to allow rain, sun, and other weather factors to fully affect them. In addition, four soil moisture sensors (Section~\ref{sec:sensors}) linked to an ESP32 are put in the soil. The setup can be seen in Figure~\ref{fig:experiment_setup}.
\begin{figure}[h]
	\centering
	\includegraphics[height=5cm]{setup.jpg}
	\caption{Experiment setup}
	\label{fig:experiment_setup}
\end{figure}
The data is gathered from the soil moisture sensor and wirelessly transmitted to a web server, where it is kept in a database with current meteorological data obtained from an API. The information gathered will subsequently be utilized to train a machine learning model that will forecast the crop's irrigation requirements. Section~\ref{sec:data_collection} explains how the data is visualized on a web page.

\paragraph{Sensors}
\label{sec:sensors}
Capacitive soil moisture sensors are a common type of sensor for measuring soil moisture levels. Capacitive sensors, as opposed to other sensors that measure moisture based on resistance, rely on the electrical capacitance of the soil, which changes with soil moisture content, allowing the sensor to measure the soil moisture level.~\parencite{sensor_types}
\newline The accuracy of capacitive soil moisture sensors is one of their key advantages. They are quite precise, even in soils with variable compositions. They are also reasonably inexpensive and simple to install, making them a viable option for sensing soil moisture in wide regions. Furthermore, they are unaffected by the presence of salts or other substances in the soil, which might impair the accuracy of other types of sensors.~\parencite{sensor_types}
\newline For these reasons, using capacitive soil moisture sensors to measure crop water needs is a good option, especially when accurate data is required in a lab setting. They provide a more accurate and consistent measurement of soil moisture levels, which is essential for effective irrigation control. Several sensors will still be used to account for the spatiotemporal variability of field-scale soil moisture.~\parencite{sensor_types, smart_irrigation_using_sensor_and_weather}
\newline For this experiment, four different sensors were used. One of them is the \textbf{SoilWatch 10 – Soil moisture sensor} from PINO-TECH, and the other three are the \textbf{Hygrometer Module V1.2} from AZ-Delivery. 
\newline The SoilWatch 10 is a sensor that can be acquired for around 25€ - 33€~\parencite{pino_tech_soilwatch_10}. It is a capacitive soil moisture sensor that can be used to measure the volumetric water content of the soil. It has a range of 0 to 50\% volumetric water content and a resolution of 0.1\% volumetric water content. It has a 3.3V to 5V operating voltage and a 0V to 3V analog output voltage. The sensor has a 3-pin interface that can be connected to a microcontroller. The sensor's pins are labeled VCC, GND, and OUT. The sensor's VCC pin is connected to the microcontroller's 3.3V pin, the GND pin is connected to the microcontroller's GND pin, and the OUT pin is connected to the microcontroller's analog input pin. 
\newline The Hygrometer Module V1.2 is a sensor that can be acquired for around 2.50€ - 5€~\parencite{az_delivery_hygrometer_module_v1_2}. It has a 5V operating voltage and the same 3-pin interface as the SoilWatch 10.

\subsubsection{Data collection and analysis}
\label{sec:data_collection}
In order to retrieve and store the sensor data, a cloud based system was developed using \gls{Nodejs}. It uses \gls{Expressjs} under the hood to expose RESTful API endpoints the ESP32 can use to send the data. The data is then stored in a \gls{Postgresql} database. The data is then visualized on a web page using \gls{Reactjs} and \gls{ReCharts}. The web page can be seen in Figure~\ref{fig:web_page}.
Each time a new sensor value is received, the backend will also query the current weather data from the \gls{Weather API} API. This data is also stored in the database.
\begin{figure}[H]
	\centering
	\includegraphics[width=6cm]{web_page.png}
	\caption{Web page}
	\label{fig:web_page}
\end{figure}
After enough data is collected, it is going to be analyzed using \gls{Python} and linear regression to train a model that will be used to forecast the crop's irrigation requirements.

\subsection{Results}
\subsubsection{Overview of the collected data}
The data collected from the sensors and the weather API is shown in Figure~\ref{fig:collected_data}. The data was collected over a period of 3 weeks. The values that have been tracked were the soil moisture, the temperature, the humidity, the UV-Index, the cloud coverage, the pressure, and the precipitation.
\begin{figure}[H]
	\centering
	\includegraphics[width=\textwidth]{collected_data.png}
	\caption{Collected data}
	\label{fig:collected_data}
\end{figure}
The data collected by sensors 2 and 4 (both being the Hygrometer Module V1.2) will not be considered as it is not accurate. After 3 and 5 days, the sensors started to show non-predictable values. This is probably due to the fact that the sensors are not waterproof and the weather was too wet. The data collected by sensors 1 and 3 is accurate and can be used for further analysis.

\subsubsection{Correlation analysis between weather data and soil moisture}
In order to determine the correlation between the weather data and the soil moisture, a correlation matrix was created. The correlation matrix can be seen in Figure~\ref{fig:correlation_matrix}. The correlation matrix shows that the soil moisture is highly correlated with the temperature and the humidity. The soil moisture is also slightly correlated with the UV-Index and the cloud coverage. The soil moisture is not correlated with the pressure.
\begin{figure}[H]
	\centering
	\includegraphics[height=9cm]{correlation_matrix.png}
	\caption{Correlation matrix}
	\label{fig:correlation_matrix}
\end{figure}
Considering the four most significant factors:
\begin{equation*}
	\begin{aligned}
		Weather\ precipitation\ (mm)              & := w_p \\
		Weather\ temperature\ (^{\circ} C)        & := w_t \\
		Weather\ humidity\ (\%)                   & := w_h \\
		Weather\ UV-Index                         & := w_u \\
		Soil\ moisture\ change\ (\frac{\%}{hour}) & := s_m \\
	\end{aligned}
\end{equation*}
The following linear regression model can be created:
\begin{equation*}
	\begin{aligned}
		\Rightarrow s_m = 0.503375w_p -0.013527w_t + 0.002103w_h -0.042218w_u -0.041001
	\end{aligned}
\end{equation*}

\subsubsection{Comparison between cloud-based irrigation scheduling and traditional manual methods}
To validate the effectiveness of the regression model and the cloud-based irrigation scheduling, a second stage of the experiment was conducted.
\newline One of the pots was watered manually every day with a fixed amount of water. The other pot was watered using the prediction algorithm based on the regression model. The results can be seen in Table \ref{table:water_measurements}.
\begin{table}[H]
	\centering
	\caption{Water Measurements}
	\begin{tabular}{|c|c|c|c|}
	  \hline
	  \multicolumn{2}{|c|}{Prediction algorithm} & \multicolumn{2}{c|}{Time based (validation)} \\
	  \hline
	  \multicolumn{1}{|c|}{Date} & \multicolumn{1}{c|}{Added Water (ml)} & \multicolumn{1}{c|}{Date} & \multicolumn{1}{c|}{Added Water (ml)} \\
	  \hline
	  09.06.2023 20:00 & 800 & 09.06.2023 20:00 & 600 \\
	  \hline
	  11.06.2023 13:00 & 800 & 10.06.2023 19:00 & 600 \\
	  \hline
	  12.06.2023 14:00 & 800 & 11.06.2023 18:00 & 600 \\
	  \hline
	  13.06.2023 18:00 & 800 & 12.06.2023 19:00 & 600 \\
	  \hline
	  14.06.2023 20:00 & 800 & 13.06.2023 18:00 & 600 \\
	  \hline
	  17.06.2023 11:00 & 800 & 14.06.2023 20:00 & 600 \\
	  \hline
	   &  & 15.06.2023 19:00 & 600 \\
	  \hline
	   &  & 16.06.2023 20:00 & 600 \\
	  \hline
	   &  & 17.06.2023 19:00 & 600 \\
	  \hline
	\end{tabular}
	\label{table:water_measurements}
  \end{table}
The results show that the prediction algorithm is more effective than the traditional manual method. The prediction algorithm was able to reduce the amount of water used by 11\% while maintaining a moisture level between 25\% and 75\%. The time scheduled method used 600ml of water every day, which kept the soil at a moisture level between 30\% and 81\%.
\newline The algorithm used would always trigger a watering event when the soil moisture level dropped below the threshold of 30\%. In reality, the real moisture value at the predicted time was lying between 25\% and 32\%.

\subsection{Discussion}
\subsubsection{Limitations of the experiment and possible improvements}
\label{sec:limitations}
There are several limitations to this experiment. The first limitation is that the experiment was conducted on a small scale. Only six salad plants were used in total, and they do not consume much water at all. The second limitation is that the experiment was conducted over a short period of time. The plants were only watered for three weeks. The third limitation is that the experiment was conducted in a controlled environment. The plants were watered with a fixed amount of water every day. In a real-world scenario, the plants would be watered with a variable amount of water every day.
\newline There are also several limitations to the regression model. One being that the experiment was conducted with a single type of plant. The results might be different for other types of plants. Also, factors like soil type and quality, growth stage, and the plant's health were not taken into account.

\subsubsection{Future directions for research}
There are several key areas that require attention in order to advance the findings of this study and explore the potential of cloud-based irrigation management systems. The recommendations that follow provide detailed insights into these future research directions.
\newline The first one is to scale up the experiment. Conducting the experiment on a larger scale is essential to validate the findings and assess the scalability of the proposed approach. This entails increasing the number of plants and expanding the study duration. By incorporating a larger sample size, the results will be more representative and provide a more robust evaluation of the water-saving potential.
\newline The next step is real-world implementation. Transitioning the experiment to a real-world scenario is crucial to understand the practical applicability of the proposed system. Conducting the research on actual fields, with varying soil types and environmental conditions, will provide valuable insights into the challenges and opportunities of implementing cloud-based irrigation management systems in agricultural practices.
\newline Also, the sensor deployment on open fields needs to be considered. Address the practical aspects of deploying sensors in open fields. Investigate the optimal placement and density of sensors to ensure representative data collection. Consider potential challenges such as protection from machinery damage, power supply, and weather resilience. Explore innovative sensor designs or protective enclosures to optimize their lifespan and minimize maintenance requirements.
\newline Evaluate the economic viability and cost-effectiveness. Conduct a thorough economic analysis to assess the cost-effectiveness and economic benefits of implementing cloud-based irrigation systems. Evaluate the return on investment for farmers, taking into account factors such as water savings, energy efficiency and potential crop yield improvements. This analysis will provide crucial insights for decision-makers and incentivize the adoption of such systems.
\newline Also consider the importance of long-term monitoring and data collection to refine the predictive algorithms over time. Continuous monitoring of soil moisture, weather conditions, and crop performance will allow for the development of adaptive algorithms that continuously improve the accuracy of irrigation predictions and recommendations.
\newline By focusing on these future research directions, it is possible to address key challenges and optimize the application of cloud-based irrigation management systems in agriculture. These advancements will contribute to sustainable water management practices, improved crop productivity, and economic benefits for farmers and the agricultural sector as a whole.

\subsection{Conclusion}
\subsubsection{Summary of the research questions and objectives}
The goal of this study was to find out if it is possible to use cloud computing and regression algorithms to predict the irrigation needs of a crop. The study also aimed to find out if it is possible to use a cloud-based system to collect and analyze the data.

\subsubsection{Contributions of the study to the field}
This study does only offer a proof of concept. It shows that it is possible to use cloud computing and regression algorithms to predict the irrigation needs of a crop. The study also shows that it is possible to use a cloud-based system to collect and analyze the data.
\newline Although this technique does not deliver perfect results, it is a very cheap and (for the farmer) easy to implement method. The costs of the sensors are relatively low, and the data can be collected and analyzed in the cloud. The farmer does not have to do anything except for installing the sensors and connecting them to the internet. In a well implemented system the farmer does not have to worry about the data collection and analysis. An automated calibration phase could be enough to make precise predictions. Also, the model could be improved by further enhance the regression model based on the data it collects while it is running.
\newline The sensors could be connected through a LoRa network to monitor the soil moisture of a whole field. The sensors could be connected to a LoRa gateway, which would then send the data to the cloud.
\newline The whole thing could possibly be sold as a service. The farmer would only have to pay a monthly fee for the service. The service provider would then take care of supplying and calibrating the right sensors. The provider would also take care of the data collection and analysis. The farmer would only have to install the sensors and connect them to the internet.

\subsection{Final remarks and recommendations}
This study shows that there is a possibility to optimize water usage by using moisture sensors and leveraging the complex work in the cloud. Still, there are many things that need to be done in order to make this a viable solution for farmers. The next step would be to conduct a long-term experiment in a real-world scenario, collect more data, and increase the variety of crops. Also, it will be determined how to calibrate the sensors and the regression model in an automated way.
\newline There are also several areas where this research could be extended. The first one is the integration of more data points. From a plant perspective, this could be crop type, crop growth stages, canopy cover, leaf area index, or plant stress indicators. Also, the soil type and its properties may be taken into account.
\newline Further research may also take into account other remote data that could give information about the field's water needs. The integration of satellite data with on-ground sensor measurements may offer broader spatial coverage and assist in scaling up the application of cloud-based irrigation management systems.
\newline As the dataset expands, more sophisticated modeling techniques, such as machine learning, may be necessary to handle the increasing complexity of the data. By analyzing large amounts of historical data, machine learning algorithms can identify complex patterns and relationships, leading to more accurate irrigation recommendations.
\newline To create a comprehensive and robust database, it is also critical to encourage data sharing and collaboration among researchers, farmers, and agricultural organizations. This collaborative effort has the potential to improve understanding of regional water management patterns, improve predictive models, and promote knowledge exchange and innovation in agricultural water-saving practices.
\newline As water saving is a global issue, it is also important to evaluate the cost-effectiveness of implementing cloud-based irrigation systems and demonstrate the potential savings in water and energy. This will help to convince governments and farmers to invest in such systems.


\newpage
\printbibliography[title=References]
\end{document}


